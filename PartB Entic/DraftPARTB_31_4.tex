\documentclass[12pt, a4papre]{article}
\usepackage[catalan]{babel}
\usepackage[unicode]{hyperref}
\usepackage{amsmath}
\usepackage{amssymb}
\usepackage{amsthm}
\usepackage{xifthen}
\usepackage{siunitx}
\usepackage{xcolor}
\usepackage{float}
\usepackage{listings}
\usepackage{setspace}
\usepackage{graphicx}
\usepackage{tikz,lipsum,lmodern}
\usepackage[most]{tcolorbox}
\usepackage{circuitikz}
\usepackage{indentfirst}
\usepackage{verbatimbox}
\usepackage[T1]{fontenc}
\usepackage{beramono}% monospaced font with bold variant
 \usepackage{tikz-timing}[2009/05/15]
\usepackage{listings}
\lstdefinelanguage{VHDL}{
   morekeywords={
     library,use,all,entity,is,port,in,out,end,architecture,of,
     begin,and
   },
   morecomment=[l]--
}
 
\usepackage{xcolor}
\colorlet{keyword}{blue!100!black!80}
\colorlet{comment}{green!50!black!90}
\lstdefinestyle{vhdl}{
   language     = VHDL,
   basicstyle   = \ttfamily,
   keywordstyle = \color{keyword}\bfseries,
   commentstyle = \color{comment}
}

\graphicspath{ {./Imatges/} }


\newcommand{\norm}[1]{\lvert #1 \rvert}

\hypersetup{
    colorlinks = true,
    linkcolor = blue
}

\author{	
		Igor Yuziv\\
		Edison Gabriel Zaragosin\\
		Oriol Torres\\
		Albert Tomas\\
		Daniel Vilardell}
		
		\title{\textbf{\textcolor{blue}{ENTIC-lab\\
	Draft PART B proposal}}}
\date{}

\begin{document}
	\maketitle

	\textbf{\textcolor{blue}{WHAT} are you going to implement?}
	
	
For part B of the project we want to add some modifications to our ROUV.
We would try to implement a sonar that would measure the distance between the ROUV and the seabed, and we would show it from a real-time graphical interface.
In addition, this information will be saved to an SD card so that we can easily check the information anytime, anywhere.
\\
	
	
	\textbf{\textcolor{blue}{HOW} are you going to build it?}
	
	
To implement the sonar, we thought of using the ultrasonic sensor called HC-SR04. With this sensor we can detect objects or the seabed at a distance between 2 and 450cm.

To implement the graphical interface we will use MATLAB where we will process the data and display it through a graph.

And finally to save the data to an SD card we will use a microSD adapter module for Arduino, and modify the code so that we can open a file, write it and save it to the external card.
\\
	


	
	\textbf{\textcolor{blue}{WHY} is it useful?}
	
The reason we want to implement a sonar is to make a study of the seabed and be able to have a tool capable of detecting any object that is polluting this environment as well as cans, bottles and plastic bags.




	

	
	
\end{document}