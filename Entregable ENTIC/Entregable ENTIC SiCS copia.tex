\documentclass[12pt, a4papre]{article}
\usepackage[catalan]{babel}
\usepackage[unicode]{hyperref}
\usepackage{amsmath}
\usepackage{amssymb}
\usepackage{amsthm}
\usepackage{xifthen}

\newcommand{\norm}[1]{\lvert #1 \rvert}

\hypersetup{
    colorlinks = true,
    linkcolor = blue
}

\author{Igor Yuziv\\
	   Daniel Vilardell}
\title{\textbf{Human Values and Professional Ethics}}
\date{}

\begin{document}
	\maketitle
	
	El camp de l’enginyeria tracta de transformacions i innovacions, per tant l’ètica constitueix un gran abast a l’hora de canviar. Per això, hem escollit aquest article per llegir-lo, parlar-ne i resumir-lo.	
	\section{Human values \& professional ethics}
	
	L’estudi de l’ètica de l’enginyeria a la universitat o a les escoles ajuda els estudiants a preparar-se i estar preparats per als grans reptes de la vida, ja que desenvolupa habilitats àmpliament aplicables en comunicació, raonament i reflexió. Aquestes habilitats potencien que els estudiants millorin amb allò que els agrada i siguin millors treballadors, sols i en grups.
	
	
	Els estàndards ètics poden influir en molts factors:
	\begin{itemize}
		\item Els dilemes ètics poden portar decisions difícils en molts projectes.
		\item Els avenços tecnològics poden requerir molta feina i paciència per estar al dia de les darreres novetats.			\item	Els valors morals tindran un paper crucial en la presa de decisions.
		\item La seguretat de les persones pot estar a les nostres mans i la nostra responsabilitat.
	\end{itemize}
	
	\section{Professional Ethics}
	
	\textbf{\textcolor{blue}{La professionalitat}} es defineix com la conducta o qualitats que caracteritzen o marquen una professió o professional. Per tant, hi ha algunes normes que un professional hauria de seguir per comportar-se adequadament en el camp de l'enginyeria. S’anomenen regles d’or i són les següents.

	\begin{itemize}
		\item \textbf{Procureu sempre l’excel·lència } :Per assolir l’èxit heu d’orientar el més alt possible.
		\item \textbf{Sigui de confiança:} Com hem dit a la secció anterior, la seguretat de les persones pot estar a les nostres mans i, per tant, la gent ha de confiar en vostè.
		\item \textbf{Sigueu responsables:} Mantingueu-vos al punt i tingueu en compte les accions que heu emprès.		\item \textbf{Sigueu respectuosos:} Per tal que les interaccions socials al lloc de treball funcionin sense problemes, eviteu conflictes i guanyeu respecte respectant els altres.
		\item \textbf{Sigueu sincers:} Aquesta és una virtut molt valorada pels empresaris i els col·legues, ja que genera confiança i augmenta el vostre valor personal per a tothom.		
		\item \textbf{Sigues competent:} Fes la teva feina correctament i intenta aprendre cada dia per millorar el teu rendiment.	
		\end{itemize}
	
	\section{Work Ethics}
	
	\textbf{Work ethics:}  L’ètica laboral està orientada a garantir l’economia, la productivitat, la seguretat, la salut i la higiene, la privadesa, la seguretat, el desenvolupament cultural i social, el benestar, el medi ambient i oferir oportunitats per a tothom, segons les seves capacitats, però sense discriminació. L’ètica professional és el conjunt d’estàndards adoptats pels professionals.
	
	\section{Valors professionals}
	
	\begin{itemize}
		\item \textbf{Integritat:} La unitat de pensament, paraula i acció i mentalitat oberta. Els ajuda a guanyar-se respecte i reconeixement per si mateixos fent la feina.		
		\item \textbf{Credibilitat  \& responsabilitat:} Treballeu de manera transparent i accepteu la responsabilitat de les vostres accions.		
		\item \textbf{Compromís:} Mantenir un interès i una fermesa sostinguts amb l’actitud fervent i l’esperança que s’aconsegueixin els objectius és compromís.		
		\item \textbf{Actitud:} L’actitud mental positiva caracteritza la fe, la integritat, l’esperança, l’optimisme, el coratge, la iniciativa, la generositat, la tolerància i el tacte.		
		\item \textbf{Valoració del temps:} El temps és un recurs rar. Aquest recurs es gasta contínuament, tant si es pren alguna decisió com si no.
	\end{itemize}
	
	\section{Resources}
	
	The article choosen can be seen in \href{https://eng.rizvi.edu.in/wp-content/uploads/2020/04/Handbook-Human-Values-and-Professional-Ethics.pdf}{Human Values and Professional Ethics}

\end{document}