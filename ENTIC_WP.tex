\documentclass[12pt, a4papre]{article}
\usepackage[catalan]{babel}
\usepackage[unicode]{hyperref}
\usepackage{amsmath}
\usepackage{amssymb}
\usepackage{amsthm}
\usepackage{xifthen}
\usepackage[dvipsnames]{xcolor}
\usepackage{indentfirst}

\newcommand{\norm}[1]{\lvert #1 \rvert}

\hypersetup{
    colorlinks = true,
    linkcolor = blue
}

\author{Components del grup}
\newpage

\title{ENTIC WP}
\date{2020}

\begin{document}
	
	\maketitle
	\newpage
	\section{WP3-ELECTRONICS}
	
	\textcolor{blue}{Question 3.1:} Which pressure increase will be observed at 3 m depth?\\

	\textcolor{blue}{Question 3.2:} Will this pressure depend or not on the atmospheric pressure at water surface? Why?\\

	\textcolor{blue}{Question 3.3:} Analyse the circuit of Fig. 4.2 and obtain the output voltage VS. Then, find the dependence of VS on the relative 	variation due to deformation (x) and on the power supply voltage (V). \\

	\textcolor{blue}{Question 3.4:} The datasheet of the sensor is available in Atenea. Identify there the full scale output, and then deduce the sensor sensitivity and its value when the supply voltage is not 10 V but 5 V.\\

	\textcolor{blue}{Question 3.5:} What output voltages V0 will provide the sensor at the water surface (at P=100 kPa) and at 3 m depth? \\

	\textcolor{blue}{Question 3.6:} In a real case and due to changing atmospheric conditions, the pressure at the water surface can be different than 100 kPa, discuss how you could fix such effect and obtain the correct pressure data. \\

	\textcolor{blue}{Question 3.7:} Analyse the circuit of Fig. 4.4 and obtain its gain expression V0/(V2-V1) when R1/R2=R4/R3.  Identify the role of Rg as a gain trimmer without compromising the R1-R4 resistance matching.\\

	\textcolor{blue}{Question 3.8:} Compare this result with the gain expression provided by the manufacturer using the resistor values shown in the INA126 datasheet.\\
	

\end{document}